
%%%%%%%%%%%%%%%%%%%%%%%%%%%% Document Setup %%%%%%%%%%%%%%%%%%%%%%%%%%%%

\documentclass[11pt]{article}
\RequirePackage[T1]{fontenc}

\usepackage{times}

% This is a helpful package that puts math inside length specifications
\usepackage{calc}

% This package helps LaTeX auto-hyphenate hyphenated words if you use
% special hyphens. For example, bio\-/mimicry will properly hyphenate
% ``mimicry'' if necessary.
%\usepackage[shortcuts]{extdash}

% Layout: Puts the section titles on left side of page
\reversemarginpar

%         PAPER SIZE, PAGE NUMBER, AND DOCUMENT LAYOUT NOTES:
%
% The next \usepackage line changes the layout for CV style section
% headings as marginal notes. It also sets up the paper size as either
% letter or A4. By default, letter was used. If A4 paper is desired,
% comment out the letterpaper lines and uncomment the a4paper lines.
%
% As you can see, the margin widths and section title widths can be
% easily adjusted.
%
% ALSO: Notice that the includefoot option can be commented OUT in order
% to put the PAGE NUMBER *IN* the bottom margin. This will make the
% effective text area larger.
%
% IF YOU WISH TO REMOVE THE ``of LASTPAGE'' next to each page number,
% see the note about the +LP and -LP lines below. Comment out the +LP
% and uncomment the -LP.
%
% IF YOU WISH TO REMOVE PAGE NUMBERS, be sure that the includefoot line
% is uncommented and ALSO uncomment the \pagestyle{empty} a few lines
% below.
%

%% Use these lines for letter-sized paper
\usepackage[paper=letterpaper,
            %includefoot, % Uncomment to put page number above margin
            marginparwidth=1.2in,     % Length of section titles
            marginparsep=.05in,       % Space between titles and text
            margin=0.75in,               % 1 inch margins
            includemp]{geometry}

%% Use these lines for A4-sized paper
%\usepackage[paper=a4paper,
%            %includefoot, % Uncomment to put page number above margin
%            marginparwidth=30.5mm,    % Length of section titles
%            marginparsep=1.5mm,       % Space between titles and text
%            margin=25mm,              % 25mm margins
%            includemp]{geometry}

%% More layout: Get rid of indenting throughout entire document
\setlength{\parindent}{0in}

% Provides special list environments and macros to create new ones
% \usepackage[shortlabels]{enumitem}
% \usepackage{etaremune}

% Simpler bibsections for CV sections
% (thanks to natbib for inspiration)
%
% * For lists of references with hanging indents and no numbers:
%
%   \begin{bibsection}
%       \item ...
%   \end{bibsection}
%
% * For numbered lists of references (with hanging indents):
%
%   \begin{bibenum}
%       \item ...
%   \end{bibenum}
%
%   Note that bibenum numbers continuously throughout. To reset the
%   counter, use
%
%   \restartlist{bibenum}
%
%   at the place where you want the numbering to reset.

\usepackage[shortlabels]{enumitem}
\usepackage{etaremune}

\makeatletter
\newlength{\bibhang}
\setlength{\bibhang}{1em}
\newlength{\bibsep}
 {\@listi \global\bibsep\itemsep \global\advance\bibsep by\parsep}
\newlist{bibenum}{enumerate}{3}
\setlist[bibenum]{label=[\arabic*],resume,leftmargin={\bibhang+\widthof{[999]}},%
        itemindent=-\bibhang,
        itemsep=\bibsep,parsep=0pt,partopsep=0pt,
        topsep=0pt}
%\let\oldendbibenum\endbibenum
%\def\endbibenum{\oldendbibenum\vspace{-.6\baselineskip}}

\newenvironment{bibenum*}
  {\renewcommand\labelenumi{[\theenumi]}%
   \etaremune[
     topsep=0pt,
     itemsep=\bibsep,
     parsep=0pt,partopsep=0pt,
     itemindent=-\bibhang,
     leftmargin={\bibhang+\widthof{[999]}}]}
  {\endetaremune}

\makeatother

%% Reference the last page in the page number
%
% NOTE: comment the +LP line and uncomment the -LP line to have page
%       numbers without the ``of ##'' last page reference)
%
% NOTE: uncomment the \pagestyle{empty} line to get rid of all page
%       numbers (make sure includefoot is commented out above)
%
\usepackage{fancyhdr,lastpage}
\pagestyle{fancy}
%\pagestyle{empty}      % Uncomment this to get rid of page numbers
\fancyhf{}\renewcommand{\headrulewidth}{0pt}
\fancyfootoffset{\marginparsep+\marginparwidth}
\newlength{\footpageshift}
\setlength{\footpageshift}
          {0.5\textwidth+0.5\marginparsep+0.5\marginparwidth-2in}
\lfoot{\hspace{\footpageshift}%
       \parbox{4in}{\, \hfill %
                    \arabic{page} of \protect\pageref*{LastPage} % +LP
%                    \arabic{page}                               % -LP
                    \hfill \,}}

% Finally, give us PDF bookmarks
\usepackage{color,hyperref}
\definecolor{darkblue}{rgb}{0.0,0.0,0.3}
\hypersetup{colorlinks,breaklinks,
            linkcolor=darkblue,urlcolor=darkblue,
            anchorcolor=darkblue,citecolor=darkblue}

%%%%%%%%%%%%%%%%%%%%%%%% End Document Setup %%%%%%%%%%%%%%%%%%%%%%%%%%%%


%%%%%%%%%%%%%%%%%%%%%%%%%%% Helper Commands %%%%%%%%%%%%%%%%%%%%%%%%%%%%

%%% HEADING AT TOP OF CURRICULUM VITAE

% The title (name) with a horizontal rule under it
% (optional argument typesets an object right-justified across from name
%  as well)
%
% Usage: \makeheading{name}
%        OR
%        \makeheading[right_object]{name}
%
% Place at top of document. It should be the first thing.
% If ``right_object'' is provided in the square-braced optional
% argument, it will be right justified on the same line as ``name'' at
% the top of the CV. For example:
%
%       \makeheading[\emph{Curriculum vitae}]{Your Name}
%
% will put an emphasized ``Curriculum vitae'' at the top of the document
% as a title. Likewise, a picture could be included:
%
%   \makeheading[{\includegraphics[height=1.5in]{my_picture}}]{Your Name}
%
% the picture will be flush right across from the name. For this example
% to work, make sure the extra set of curly braces is included. Also
% makes ure that \usepackage{graphicx} is somewhere in the preamble.
\newcommand{\makeheading}[2][]%
        {\hspace*{-\marginparsep minus \marginparwidth}%
         \begin{minipage}[t]{\textwidth+\marginparwidth+\marginparsep}%
             {\large \bfseries #2 \hfill #1}\\[-0.15\baselineskip]%
                 \rule{\columnwidth}{1pt}%
         \end{minipage}}

%%% SECTION HEADINGS

% The section headings. Flush left in small caps down pseudo-margin.
%
% Usage: \section{section name}
\renewcommand{\section}[1]{\pagebreak[3]%
    \vspace{1.3\baselineskip}%
    \phantomsection\addcontentsline{toc}{section}{#1}%
    \noindent\llap{\scshape\smash{\parbox[t]{\marginparwidth}{\hyphenpenalty=10000\raggedright #1}}}%
    \vspace{-\baselineskip}\par}

%%% LISTS

% This macro alters a list by removing some of the space that follows the list
% (is used by lists below)
\newcommand*\fixendlist[1]{%
    \expandafter\let\csname preFixEndListend#1\expandafter\endcsname\csname end#1\endcsname
    \expandafter\def\csname end#1\endcsname{\csname preFixEndListend#1\endcsname\vspace{-0.6\baselineskip}}}

% These macros help ensure that items in outer-type lists do not get
% separated from the next line by a page break
% (they are used by lists below)
\let\originalItem\item
\newcommand*\fixouterlist[1]{%
    \expandafter\let\csname preFixOuterList#1\expandafter\endcsname\csname #1\endcsname
    \expandafter\def\csname #1\endcsname{\let\oldItem\item\def\item{\pagebreak[2]\oldItem}\csname preFixOuterList#1\endcsname}
    \expandafter\let\csname preFixOuterListend#1\expandafter\endcsname\csname end#1\endcsname
    \expandafter\def\csname end#1\endcsname{\let\item\oldItem\csname preFixOuterListend#1\endcsname}}
\newcommand*\fixinnerlist[1]{%
    \expandafter\let\csname preFixInnerList#1\expandafter\endcsname\csname #1\endcsname
    \expandafter\def\csname #1\endcsname{\let\oldItem\item\let\item\originalItem\csname preFixInnerList#1\endcsname}
    \expandafter\let\csname preFixInnerListend#1\expandafter\endcsname\csname end#1\endcsname
    \expandafter\def\csname end#1\endcsname{\csname preFixInnerListend#1\endcsname\let\item\oldItem}}

% An itemize-style list with lots of space between items
%
% Usage:
%   \begin{outerlist}
%       \item ...    % (or \item[] for no bullet)
%   \end{outerlist}
\newlist{outerlist}{itemize}{3}
    \setlist[outerlist]{label=\enskip\textbullet,leftmargin=*}
    \fixendlist{outerlist}
    \fixouterlist{outerlist}

% An environment IDENTICAL to outerlist that has better pre-list spacing
% when used as the first thing in a \section
%
% Usage:
%   \begin{lonelist}
%       \item ...    % (or \item[] for no bullet)
%   \end{lonelist}
\newlist{lonelist}{itemize}{3}
    \setlist[lonelist]{label=\enskip\textbullet,leftmargin=*,partopsep=0pt,topsep=0pt}
    \fixendlist{lonelist}
    \fixouterlist{lonelist}

% An itemize-style list with little space between items
%
% Usage:
%   \begin{innerlist}
%       \item ...    % (or \item[] for no bullet)
%   \end{innerlist}
\newlist{innerlist}{itemize}{3}
    \setlist[innerlist]{label=\enskip\textbullet,leftmargin=*,parsep=0pt,itemsep=0pt,topsep=0pt,partopsep=0pt}
    \fixinnerlist{innerlist}

% An environment IDENTICAL to innerlist that has better pre-list spacing
% when used as the first thing in a \section
%
% Usage:
%   \begin{loneinnerlist}
%       \item ...    % (or \item[] for no bullet)
%   \end{loneinnerlist}
\newlist{loneinnerlist}{itemize}{3}
    \setlist[loneinnerlist]{label=\enskip\textbullet,leftmargin=*,parsep=0pt,itemsep=0pt,topsep=0pt,partopsep=0pt}
    \fixendlist{loneinnerlist}
    \fixinnerlist{loneinnerlist}

%%% EXTRA SPACE

% To add some paragraph space between lines.
% This also tells LaTeX to preferably break a page on one of these gaps
% if there is a needed pagebreak nearby.
\newcommand{\blankline}{\quad\pagebreak[3]}
\newcommand{\halfblankline}{\quad\vspace{-0.5\baselineskip}\pagebreak[3]}

%%% FORMATTING MACROS

% Provides a linked \doi{#1} that links doi:#1 to http://dx.doi.org/#1
\usepackage{doi}
% To change the text before the DOI, adjust this command
%\renewcommand\doitext{doi:}

% Provides a linked \url{#1} that doesn't require escape characters
\usepackage{url}

% You can adjust the style \url{} uses here:
% (options are: same, rm, sf, tt; defaults to tt)
\urlstyle{same}

% For \email{ADDRESS}, links ADDRESS to the url mailto:ADDRESS
% (uncomment to typeset the e\-/mail address in typewriter font;
%  otherwise, will be typeset in the \urlstyle above)
%\DeclareUrlCommand\emaillink{\urlstyle{tt}}
\providecommand*\emaillink[1]{\nolinkurl{#1}}
\providecommand*\email[1]{\href{mailto:#1}{\emaillink{#1}}}

\providecommand\BibTeX{{B\kern-.05em{\sc i\kern-.025em b}\kern-.08em \TeX}}
\providecommand\Matlab{\textsc{Matlab}}

% Custom hyphenation rules for words that LaTeX has trouble with
\hyphenation{bio-mim-ic-ry bio-in-spi-ra-tion re-us-a-ble pro-vid-er Media-Wiki}

%%%%%%%%%%%%%%%%%%%%%%%% End Helper Commands %%%%%%%%%%%%%%%%%%%%%%%%%%%

%%%%%%%%%%%%%%%%%%%%%%%%% Begin CV Document %%%%%%%%%%%%%%%%%%%%%%%%%%%%

\begin{document}
\makeheading{Joseph Stachelek}

\section{Contact Information}

% NOTE: Mind where the & separators and \\ breaks are in the following
%       table. Table is one row made up of three parboxes. The left
%       parbox has address info, the middle parbox has a vertical bar,
%       and the right parbox has phone and electronic contact
%       information.
%
% MACROS: \rcollength is the width of the right column of the table
%             (adjust it to your liking; default is 1.85in).
%         \spacewidth is width of area between left and right boxes.
%
\newlength{\rcollength}\setlength{\rcollength}{1.85in}%
\newlength{\spacewidth}\setlength{\spacewidth}{20pt}
%
\begin{tabular}[t]{@{}p{\textwidth-\rcollength-\spacewidth}@{}p{\spacewidth}@{}p{\rcollength}}%

% Address box
\parbox{\textwidth-\rcollength-\spacewidth}{%
\href{https://limnology.wisc.edu/}{Center for Limnology}\\
\href{https://www.wisc.edu/}{University of Wisconsin}\\
680 North Park Street\\
Madison, WI, 53706, USA}

&
% Uncomment to add a vertical bar in middle of contact information
%{\vrule width 0.5pt}
\parbox[m][5\baselineskip]{\spacewidth}{} &

% Non-snail-mail contact information
\parbox{\rcollength}{%
\vspace{-1em}
\textit{Telephone:} +1-517-884-1769 \\
\textit{E-mail:} \email{stachelek@wisc.edu}\\
\textit{WWW:} \href{https://jsta.rbind.io/}{https://jsta.rbind.io}}
\textit{orcid:} \href{http://orcid.org/0000-0002-5924-2464}{0000-0002-5924-2464}

\end{tabular}

%%
%% In modern CV's, it seems like ``Objective'' is frowned upon. Instead,
%% incorporate it into a well-constructed cover letter. The ``More
%% information'' can go at the end of the CV, but it should not distract
%% from the section giving references available to contact.
%%
%
% \section{Objective}
%
% Placement in an academic position (i.e., faculty, postdoctoral, or
% research scientist) that allows for advanced research in distributed
% complex adaptive systems (i.e., modeling, analysis, design, and
% verification) with a particular focus on the control of engineered
% agents (e.g., for communications, control, software, electronics, and
% sustainability) and the analysis of biological phenomena (e.g.,
% self-organization, ecological rationality)
% \begin{innerlist}
% \item More information and auxiliary documents can be found at\\\url{http://www.tedpavlic.com/facjobsearch/}
% \end{innerlist}

% \section{Research Interests}
% 
% Landscape Ecology, GIS, Open Science, Sea Level Rise
\vspace{-1em}
\section{Education} \vspace{-1em} \begin{outerlist}
    \item[] PhD, Fisheries and Wildlife, \href{http://www.fw.msu.edu/}{\textbf{Michigan State University}}%
            \hfill \textbf{2016 to 2020} \hspace{1em}
    \end{outerlist}
    
\begin{outerlist}
    \item[] MS, Marine Science, \href{http://www.utexas.edu/}{\textbf{University of Texas at Austin}}%
            \hfill \textbf{2009 to 2012} \hspace{1em}
    \end{outerlist}
    
\begin{outerlist}
    \item[] BS, Marine Science, \href{http://www.maine.edu/}{\textbf{University of Maine}}%
            \hfill \textbf{2004 to 2008} \hspace{1em}
    \end{outerlist}

\section{Professional Experience} \vspace{-1em} 
\begin{outerlist}
    \item[] Post-doctoral Research Associate, \href{https://limnology.wisc.edu/}{\textbf{University of Wisconsin,}} \hfill \textbf{2020 to Present}
    \vspace{-0.6em}
    \item[] \href{https://limnology.wisc.edu/}{\textbf{Center for Limnology}}%
\end{outerlist}
\vspace{-0.5em}

\begin{outerlist}
    \item[] Research Scientist, \href{http://www.sfwmd.gov/}{\textbf{South Florida Water Management District}} \hfill \textbf{2012 to 2016}
    \vspace{-0.4em}
    % \item[] \textit{Water quality modelling, data curation, and GIS tool development.}%
\end{outerlist}

\begin{outerlist}
    \item[] Research Technician, \href{http://www.utmsi.utexas.edu/staff/dunton/}{\textbf{University of Texas Marine Botany Lab}} \hfill \textbf{2008 to 2009}
    \vspace{-0.4em}
    % \item[] \textit{Laboratory nutrient assays, wetland surveys, and database management.}%
            
\end{outerlist}

\section{Refereed Publications}

\begin{bibenum*}

\item Hanson, P., Stillman, A.B., Jia, X., Karpatne, A., Dugan, H.A., Carey, C.C., \textbf{Stachelek, J.}, Ward, N.K, Zhang, Y., Read, J.S., Kumar, V., 2020. Predicting lake surface water phosphorus dynamics using process-guided machine learning. \emph{Ecological Modelling}, 430, 1-11. \href{https://doi.org/10.1016/j.ecolmodel.2020.109136}{10.1016/j.ecolmodel.2020.109136}.

\item \textbf{Stachelek, J.}, Weng, W., Carey, C.C., Kemanian, A.R., Cobourn, K.M., Wagner, T., Weathers, K.C., Soranno, P.A., 2020. Granular measures of agricultural land-use influence lake nitrogen and phosphorus differently at macroscales. \emph{Ecological Applications}, 30(8) 1-13. \href{https://doi.org/10.1002/eap.2187}{10.1002/eap.2187}.

\item Servais, S., Kominoski, J.S., Coronado-Molina, C., Bauman, L., Davis, S.E., Gaiser, E.E., Kelly, S., Madden, C., Mazzei, V., Rudnik, D., Santamaria, F., Sklar, F.H., \textbf{Stachelek, J.}, Troxler, T.G., Wilson, B.J., 2020. Effects of saltwater pulses on soil microbial enzymes and organic matter breakdown in freshwater and brackish coastal wetlands. \emph{Estuaries and Coasts}, 43, 814-830. \href{https://doi.org/10.1007/s12237-020-00708-1}{10.1007/s12237-020-00708-1}.

\item Soranno, P.A., Cheruvelil, K.S., Liu, B., Wang, Q., Tan, P.N., Zhou, J., King, K.B.S., McCullough, I.M., \textbf{Stachelek, J.}, Bartley, M., Filstrup, C.T., Hanks, E.M., Lapierre, J.F., Lottig, N.R., Schliep, E.M., Wagner, T., Webster, K.E., 2020. Ecological prediction at macroscales using big data: Does sampling design matter? \emph{Ecological Applications}, 30(6), 1-13. \href{https://doi.org/10.1002/eap.2123}{10.1002/eap.2123}.

\item Wagner, T., Lottig, N., Bartley, M.L., Hanks, E.M., Schliep, E.M., Wikle, N.B., King, K.B.S., McCullough, I., \textbf{Stachelek, J.}, Cheruvelil, K.S., Filstrup, C.T., Lapierre, J.F., Liu, B., Soranno, P.A., Tan, P.N., Wang, Q., Webster, K.,  Zhou, J. 2019. Increasing accuracy of lake nutrient predictions in thousands of lakes by leveraging water clarity data. \emph{Limnology and Oceanography: Letters}, 5, 228-235. \href{https://doi.org/10.1002/lol2.10134}{10.1002/lol2.10134}. 

\item McCullough, I., King K., \textbf{Stachelek, J.}, Diaz, J., Soranno, P.A., Cheruvelil K.S. 2019. Applying the patch-matrix model to lakes: a connectivity-based conservation framework. \emph{Landscape Ecology}, 34, 2703-2718. \href{https://doi.org/10.1007/s10980-019-00915-7}{10.1007/s10980-019-00915-7}.

\item Qian, S.S., Stow, C.A., Nojavan, F., \textbf{Stachelek, J.}, Cha, Y., Alameddine, I., Soranno, P.A. 2019. The Implications of Simpson's Paradox for Cross-Scale Inference Among Lakes. \emph{Water Research}, 163, 1-7. \href{https://doi.org/10.1016/j.watres.2019.114855}{10.1016/j.watres.2019.114855}.

\item McCullough, I., Cheruvelil, K.S, Lapierre, J.F., Lottig, N., Moritz, M., \textbf{Stachelek, J.}, Soranno, P.A. 2019. Do lakes feel the burn? Ecological consequences of increasing exposure of lakes to fire in the continental United States. \emph{Global Change Biology}, 25(9), 2841-2854. \href{https://doi.org/10.1111/gcb.14732}{10.1111/gcb.14732}.

\item Collins, S.M., Yuan, S., Tan, P.N., Oliver, S.K., Lapierre, J.F., Cheruvelil, K.S., Fergus, C.E., Skaff, N.K., \textbf{Stachelek, J.}, Wagner, T., Soranno, P.A. 2019. Winter Precipitation and Summer Temperature Predict Lake Water Quality at Macroscales. \emph{Water Resources Research}, 55(4), 2708-2721. \href{https://doi.org/10.1029/2018WR023088}{10.1029/2018WR023088}.

\item \textbf{Stachelek, J.}, Soranno, P.A. 2019. Does freshwater connectivity influence phosphorus retention in lakes? \emph{Limnology and Oceanography}, 64(4), 1586-1599. \href{https://doi.org/10.1002/lno.11137}{10.1002/lno.11137}.

\item Ward, N.K., Fitchett, L., Hart, J.A., Shu, L., \textbf{Stachelek, J.}, Weng, W., Zhang, Y., Dugan, H., Hetherington, A., Boyle, K., Carey, C.C., Cobourn, K.M., Hanson, P.C., Kemanian, A.R., Sorice, M.G., Weathers, K.C. 2018. Integrating fast and slow processes is essential for simulating human-freshwater interactions. \emph{Ambio}, 48(10), 1169-1182 \href{https://doi.org/10.1007/s13280-018-1136-6}{10.1007/s13280-018-1136-6}.

\item Wilson, B.J., Servais, S., Mazzei, V., Kominoski, J.S., Hu, M., Davis, S.E., Gaiser, E., Sklar, F., Bauman, L., Kelly, S., Madden, C., Richards, J., Rudnick, D., \textbf{Stachelek J.}, Troxler, T. 2018. Salinity pulses interact with seasonal dry‐down to increase ecosystem carbon loss in marshes of the Florida Everglades. \emph{Ecological Applications}, 28(8), 2092-2108. \href{https://doi.org/10.1002/eap.1798}{10.1002/eap.1798}.

\item Mazzei, V., Gaiser, E., Kominoski, J., Wilson, B.J., Servais, S., Bauman, L., Davis, S.E., Kelly, S., Sklar, F.H., Rudnick, D.T., \textbf{Stachelek J.}, Troxler, T. 2018. Functional and compositional responses of periphyton mats to simulated saltwater intrusion in the southern Everglades. \emph{Estuaries and Coasts}, 41(7), 2105-2119. \href{https://doi.org/10.1007/s12237-018-0415-6}{10.1007/s12237-018-0415-6}.

\item \textbf{Stachelek J.}, Kelly, S.P, Sklar, F., Coronado, C.M., Troxler, T., Bauman, L. 2018. In-situ simulation of sea-level rise impacts on coastal wetlands using a flow-through mesocosm approach. \emph{Methods in Ecology and Evolution}, 9(8), 1908-1915. \href{https://doi.org/10.1111/2041-210X.13028}{10.1111/2041-210X.13028}.

\item Cobourn, K.M., Carey, C.C., Boyle, K.J., Duffy, C., Dugan, H.A., Farrell, K.J., Fitchett, L., Hanson, P.C., Hart, J.A., Henson, V.R., Hetherington, A.L., Kemanian, A.R., Rudstam, L.G., Shu, L., Soranno, P.A., Sorice, M.G., \textbf{Stachelek J.}, Ward, N.K., Weathers, K.C., Weng, W., Zhang, Y. 2018. From concept to practice to policy: modeling coupled natural and human systems in lake catchments. \emph{Ecosphere}, 9(5):e02209 \href{https://doi.org/10.1002/ecs2.2209}{10.1002/ecs2.2209}.

\item \textbf{Stachelek J.}, Ford, C., Kincaid, D., King, K., Miller, H, and Nagelkirk, R. 2018. The National Eutrophication Survey: lake characteristics and historical nutrient concentrations. \emph{Earth Syst. Sci. Data}, 10, 81-86. \href{https://doi.org/10.5194/essd-10-81-2018}{10.5194/essd-10-81-2018}.

\item Rougier NP, Hinsen K, <34 alphabetical authors>, \textbf{Stachelek J}, <8 alphabetical authors>. 2017. Sustainable computational science: the ReScience initiative. \emph{PeerJ Computer Science}. 3:e142 \href{https://doi.org/10.7717/peerj-cs.142}{10.7717/peerj-cs.142}.

\item Soranno P.A., <62 alphabetical authors>, \textbf{Stachelek J}, <15 alphabetical authors>. 2017. LAGOS-NE: A multi-scaled geospatial and temporal database of lake ecological context and water quality for thousands of U.S. lakes. \emph{Gigascience}. 6(12). \href{https://doi.org/10.1093/gigascience/gix101}{10.1093/gigascience/gix101}

\item Hollister J and \textbf{Stachelek J} 2017. lakemorpho: Calculating lake morphometry metrics in R [version 1; referees: 2 approved]. \emph{F1000Research}. 6:1718. \href{https://doi.org/10.12688/f1000research.12512.1}{10.12688/f1000research.12512.1}.

\item \textbf{Stachelek J} 2016. [Re] Least-cost modelling on irregular landscape graphs. \emph{ReScience}. 2(1): 1-4. \href{https://github.com/ReScience-Archives/Stachelek-2016/raw/master/article/article.pdf}{10.5281/zenodo.47146}.

\item \textbf{Stachelek J} and Madden C.J. 2015. Application of Inverse Path Distance Weighting for high density spatial mapping of coastal water quality patterns. \emph{International Journal of Geographical Information Science}. 29(7), 1240-1250. \href{https://doi.org/10.1080/13658816.2015.1018833}{10.1080/13658816.2015.1018833}.

\item \textbf{Stachelek J} and Dunton, K.H. 2013. Freshwater inflow requirements for the Nueces Delta, Texas: Spartina alterniflora as an indicator of ecosystem condition. \emph{Texas Water Journal}. 4(2), pp.62-73. \href{https://journals.tdl.org/twj/index.php/twj/article/view/6354}{https://journals.tdl.org/twj/index.php/twj/article/view/6354}.


\end{bibenum*}

% Add a little space to nudge next ``Conference Publications'' marginpar
% down to make room for tall ``Submitted Conference Publications''
% marginpar. If there are enough submitted journal publications, this
% space will not be needed (and should be removed).
\vspace{0.1in}

% \section{Publications In-Review (available upon request)}
% 
% \restartlist{bibenum}
% 
% \begin{bibenum*}
% 
%   \item Park, S.R., \textbf{Stachelek, J.}, Dunton K.H. Photosynthesis and drought resilience in three emergent vascular plant species common to marshes of the western Gulf of Mexico.
%    
%     \item \textbf{Stachelek J.}, Madden, C.J., Kelly, S.P, Blaha, M. Fine-scale relationships between phytoplankton abundance and environmental drivers in Florida Bay, USA. \emph{Estuarine, Coastal, and Shelf Science}.
%    
% \end{bibenum*}

\section{Public Datasets}

\restartlist{bibenum}
\begin{bibenum*}

\item \textbf{Stachelek J.} 2019. Freshwater connectivity and stream morphology metrics for Northeast and Midwestern US lakes. \href{https://doi.org/10.5281/zenodo.2554212}{https://doi.org/10.5281/zenodo.2554212}

\item \textbf{Stachelek J.}, Ford C., Kincaid D., King K., Miller H., Nagelkirk R. 2017. The National Eutrophication Survey: lake characteristics and historical nutrient concentrations. \emph{KNB Data Repository} \href{http://dx.doi.org/10.5063/F10G3H3Z}{http://dx.doi.org/10.5063/F10G3H3Z}

\item Madden C, \textbf{Stachelek J}, Kelly S, Blaha M. 2017. Florida Bay water quality estimated by underway flow-through measurement. \emph{KNB Data Repository} \\ \href{http://dx.doi.org/10.5063/F11R6NGR}{http://dx.doi.org/10.5063/F11R6NGR}
   
\end{bibenum*}

\section{\href{http://github.com/jsta}{Software and Package Development} \tiny{(See \underline{\href{https://jsta.github.io/gh_cran_portfolio/}{Data Science Portfolio}}})}

\restartlist{bibenum}
\begin{bibenum*}

\item \textbf{Stachelek, J.} 2019. gssurgo: Python toolbox enabling an open source gSSURGO workflow. \href{https://github.com/jsta/gssurgo}{Python toolbox}.
\item \textbf{Stachelek, J.} 2019. nhdR: Tools for working with the National Hydrography Dataset. \href{https://github.com/jsta/nhdR}{R package}.
\item \textbf{Stachelek J.}, Oliver S. 2019. LAGOSNE: Interface to the Lake Multi-scaled Geospatial and Temporal Database. \href{https://cran.r-project.org/package=LAGOSNE}{R package}.
\item \textbf{Stachelek J.} 2019. dbhydroR: Everglades Hydrologic and Water Quality Data from R. \href{https://github.com/ropensci/dbhydroR}{R package}.

\item Chamberlain, S., Anderson, B., Salmon, M., Erickson, A., Potter, N., \textbf{Stachelek, J.}, Simmons, A., Ram, K., Hart, E. 2019. rnoaa: 'NOAA' Weather Data from R. \href{https://github.com/ropensci/rnoaa}{R package}.

\item \textbf{Stachelek J.} 2018. wikilake: Scrape Lake Metadata Tables from Wikipedia. \href{https://github.com/jsta/wikilake}{R package}.

\item \textbf{Stachelek J.} 2018. ipdw: Interpolation by Inverse Path Distance Weighting. \href{https://github.com/jsta/ipdw}{R package}.

\item Hollister, J.W., \textbf{Stachelek J.} 2018. lakemorpho: Lake Morphometry Metrics in R. \href{https://github.com/jhollist/lakemorpho}{R package}.
\end{bibenum*}

\section{Conference Presentations}

\restartlist{bibenum}
\begin{bibenum*}

\item Ladwig, R., L. Gao, J. Willard, A. Appling, A. Delany, \textbf{J. Stachelek}, H.A. Dugan, S. Oliver, J.S. Read, P.C. Hanson. Two-layer bayesian dissolved oxygen model for ecological process discovery. Poster Presentation at the Global Lake Ecological Observatory Network Meeting, Virtual, Oct 19 2020.

\item Weng, W., K. M. Cobourn, A. R. Kemanian, K. J. Boyle, Y. Shi, \textbf{J. Stachelek}, C. White. Quantifying Co-benefits of Water Quality Policies:  An Integrated Assessment Model of Nitrogen Management. Selected paper, Agricultural and Applied Economics Association Meeting, Aug 11 2020.

\item \textbf{Stachelek, J.}, Rodriguez L., Soranno P.A. 2019. Spatial patterning and prediction of lake depth at continental scales. Poster Presentation at the Global Lake Ecological Observatory Network Meeting, Huntsville, ON, CA. Nov 6 2019.

\item \textbf{Stachelek, J.}, <8 authors>. 2019. \href{https://doi.org/10.6084/m9.figshare.12486164.v2}{Analysis of 500 lake catchments reveals the relationship between crop type, fertilizer and manure inputs and lake nutrient concentrations}. Oral Presentation at the Ecological Society of America Meeting, Louisville, Kentuckey, USA. Aug 13 2019.

\item McCullough, I.M., King, K., \textbf{Stachelek, J.}, <3 authors>. 2019. No lake left behind: Do protected areas facilitate biological connectivity among lakes? Oral Presentation at the Ecological Society of America Meeting, Louisville, Kentuckey, USA. Aug 13 2019.

\item Kelly, S.P., \textbf{Stachelek, J.}, Strazisar, T. 2019. Examining the effects of sea-level rise on Everglades coastal marshes using coupled mesocosm and in-situ field manipulations: design and implementation. Oral Presentation to the Greater Everglades Ecosystem Restoration Conference, Coral Springs, Florida, USA. Apr 2019.

  \item \textbf{Stachelek, J.}, Soranno, P.A. 2018. \href{https://doi.org/10.6084/m9.figshare.5903875.v2}{Does Lake and Stream Connectivity Control Phosphorus Retention in Lakes?} Oral Presentation to the Association for the Sciences of Limnology and Oceanography, Victoria, British Columbia, CA. Jun 15 2018.

  \item Smith, N., Soranno, P.A., Cheruveill, K., Gries, C., \textbf{Stachelek, J.} 2018. Mapping a Journey Towards Open Science: Lessons Learned Building a Lake Water Quality Geodatabase. Oral Presentation at the American Water Resources Association GIS \& Water Resources X Conference, Orlando, Florida, USA. Apr 23 2018.

  \item \textbf{Stachelek, J.}, Soranno, P.A. 2017. \href{https://doi.org/10.6084/m9.figshare.9638735.v1}{Does Connectivity Control Lake Phosphorus Retention?} Poster Presentation at the Global Lake Ecological Observatory Network Meeting, New Paltz, New York, USA. Nov 29 2017.

  \item Collins, S.M, Cheruvelil, K.S., Fergus, C.E., Lapierre, J.F., Oliver, S.K., Scott, C.E., Skaff, N.K., Soranno, P.A., \textbf{Stachelek, J.}, Tan, P., Yuan, S. and Wagner, T. 2017. Which measures of climate are the best predictors of lake water quality at sub-continental scales? Oral Presentation at the Ecological Society of America Meeting, Portland, Oregon, USA. Aug 06 2017.

  \item Nowosad, J., Teucher A., \textbf{Stachelek, J.}, Cotton, R., Vitolo, C. 2017. The State of Data on CRAN: Discovering Good Data Packages. Oral Presentation at the rOpenSci Unconference, Los Angeles, CA, USA. May 26 2017.

  \item Madden, C.J., M. Koch, T. Strazisar, T. Troxler, Y. Shangguan, S.P. Kelly, \textbf{J. Stachelek}, R.M. Price, and F.H. Sklar, 2017. Interconnections in the Everglades and Florida Bay Watershed: Implications for Ecosystem Integrity. 2017 American Water Resources Association Spring Conference, Snowbird, Utah, May 3, 2017.

  \item Sklar, F.H., C. Coronado-Molina, \textbf{J. Stachelek}, S.P. Kelly, and T. Troxler, 2017. Coastal Subsidence as a Function of Salinity Intrusion and Peat Decomposition in a Karst Environment. Greater Everglades Ecosystem Restoration Meeting, Coral Springs, Florida, April 20, 2017.

  \item \textbf{Stachelek, J.} 2017. \href{https://doi.org/10.6084/m9.figshare.8187038.v1}{Lake Connectivity Effects on Phosphorus in 1,000s of Lakes}. Oral Presentation at the Michigan State University Fisheries and Wildlife Graduate Research Symposium. Feb 24 2017.

  \item Kominoski, J, Evelyn Gaiser, Tiffany Troxler, <12 others>, \textbf{Stachelek, J.}, <20 others>. 2016. Saltwater intrustion and carbon loss: identifying the biogeochemical attributes that drive differential responses among coastal wetlands. Poster Presentation at the International Long Term Ecological Research Meeting, Kruger National Park, South Africa. Oct 09 2016.

  \item \textbf{Stachelek, J.}, Madden, C.J., Kelly, S., Blaha, M. 2016. \href{http://doi.org/10.6084/m9.figshare.2775322.v4}{Fine-scale spatial patterning of phytoplankton abundance in a coastal estuary}. Oral Presetentation at the Ecological Society of America Meeting, Fort Lauderdale, Florida, USA. Aug 07 2016. 
  
    \item Troxler, T.G., <12 others>, \textbf{Stachelek, J.}, Wilson, B.J. 2016. Carbon cycle science in the Florida Coastal Everglades: Research to inform carbon and water management. Oral Presentation at the Ecological Society of America Meeting, Fort Lauderdale, Florida, USA. Aug 07 2016.
%   
    \item Sklar, F.H., Coronado, C., Troxler, T.G., \textbf{Stachelek, J.}, Kelly, S., Kominoski, J.S. 2016. Coastal subsidence as a function of salinity intrusion and peat decomposition in a karst environment. Oral Presentation at the Ecological Society of America Meeting, Fort Lauderdale, Florida, USA. Aug 07 2016.
%   
    \item \textbf{Stachelek, J.}. 2015. Resolving Fine-Scale Patterning and Restoration Outcomes in the Coastal Everglades. Oral Presentation at the Greater Everglades Ecosystem Restoration Meeting, Coral Springs, Florida, USA. Apr 21 2015. %[slides](http://conference.ifas.ufl.edu/geer2015/Documents/Speaker%20Presentations/SESSION%2010/1400_Stachelek_Joseph.pdf) 
%   
    \item Kominoski, J., Servais, S., B.J. Wilson, V. Mazzei, E.E. Gaiser, T. Troxler, C. Coronado-Molina, S.E. Davis, S.P. Kelly, \textbf{J. Stachelek}, F.H. Sklar, C.J. Madden, and L. Bauman. 2015. Effects of increased water salinity and inundation on microbial processing of carbon and nutrients in oligohaline wetland soils. Oral Presentation at the Ecological Society of America 100th Annual Meeting, Baltimore, Maryland, USA. Aug 09 2015.
%   
    \item Troxler, T., F.H. Sklar, S.E. Davis, E.E. Gaiser, S.P. Kelly, J. Kominoski, C.J. Madden, V. Mazzei, C. Coronado-Molina, D.T. Rudnick, S. Servais, \textbf{J. Stachelek}, and B.J. Wilson. 2015. The effects of projected sea-level rise on Everglades coastal ecosystems: Evaluating the potential for and mechanisms of peat collapse. Oral Presentation at the Ecological Society of America 100th Annual Meeting, Baltimore, Maryland, USA. Aug 09 2015.
%   
    \item Wilson, B., Troxler, T., Gaiser, E., Kominoski, J., Richards, J., Servais, S., \textbf{Stachelek, J.}, Kelly, S. Kelly, Sklar, F., Coronado-Molina, C., Madden, C., Davis, S.E., Mazzei, V., Schulte, N., Bauman, L. 2014. Ecosystem Productivity Responses to Saltwater Intrusion and P Loading As a Result of Future Sea Level Rise in the Coastal Everglades. Poster Presentation at the American Geophysical Union Meeting, San Fransisco, California, USA. Dec 15 2014.
   
    \item \textbf{Stachelek, J.}, Madden, C.J. 2013. High Density Spatial Mapping of Water Quality Patterns Reveals Impacts of Freshwater Inputs in Florida Bay, USA. Poster Presentation at the Coastal and Esutarine Reserach Federation, San Diego, California, USA. Nov 03 2013.
   
    \item Madden, C.J., McDonald, A.A., Koch-Rose, M., Glibert, P., Kelly, S.P., \textbf{Stachelek, J.} 2013. Exploring Linkages Among Watershed-Estuary Processes in the Southern Everglades, Florida Bay Using Model Synthesis. Oral Presentation at the Coastal and Esutarine Reserach Federation, San Diego, California, USA. Nov 03 2013.
   
   \item \textbf{Stachelek J.}, Dunton K.H. 2012. \href{http://doi.org/10.6084/m9.figshare.6614960.v1}{Porewater salinity dynamics in an irregularly flooded marsh}. Poster Presentation at the Texas Bays and Estuaries Meeting, Port Aransas, Texas, USA. April 2012.
   
    \item \textbf{Stachelek J.}, Dunton, K.H. 2011. Estimation of freshwater inflow requirements for a semi-arid salt marsh using emergent plants as indicators of ecosystem condition. Oral Presentation at the Coastal and Esutarine Reserach Federation, Daytona Beach, Florida, USA. Nov 06 2011.

    \item Park, S.R., \textbf{Stachelek J.}, Dunton, K.H. 2011. Seasonal variations in photosynthetic characteristics of three major emergent salt marsh plants in the Southwestern Gulf of Mexico. Oral Presentation at the Coastal and Esutarine Reserach Federation, Daytona Beach, Florida, USA. Nov 06 2011.
 
    \item \textbf{Stachelek J.}, Dunton, K.H. 2011. Porewater salinity dynamics within emergent salt marsh vegetation. Oral Presentation at the Benthic Ecology Meeting, Mobile, Alabama, USA. Mar 16 2011.
    
    \item \textbf{Stachelek, J.}, Brawley, S.H. 2008. Constructing a guide to intertidal algae of Acadia and testing DNA barcoding. Poster Presentation at the Northeast Algal Society Meeting, Durham, New Hampshire, USA. Apr 18 2008.

\end{bibenum*}

\section{Invited Presentations}

\restartlist{bibenum}
\begin{bibenum*}

\item \textbf{Stachelek, J.}, Scaling and extrapolation of lake water quality using abstraction, Center for Limnology, University of Wisconsin, Oct 7, 2020

\end{bibenum*}

\section{Workshops Given} % \vspace{-1em}

\href{}{\textbf{Miscellaneous}}
\begin{outerlist}

\item[] 
    \begin{innerlist}
    \item Version control software (Git) for application in academic research. Oct 13 2020. Global Lake Ecological Observatory Network (GLEON) Meeting.
        \item Basics of geospatial analysis for research computing. Feb 27 2020, Great Lakes Acoustic Telemetry Observation System (GLATOS) Coordination Meeting.
        \item Basics of editing Wikipedia, recommended practices, and reasonable workflows. Jun 12 2018, Association for the Sciences of Limnology and Oceanography (ASLO) Meeting.
        \item Version control software (Git) for application in academic research. Apr 28 2017. MSU EEBB Programming Group.
        \item Basics of Python and the Linux command line for research computing. Jan 12 2017. MSU Institute for Cyber-enabled Research.
    \end{innerlist}

\end{outerlist}

\halfblankline

\href{http://software-carpentry.org/}{\textbf{Software Carpentry (certified instructor)}}
\begin{outerlist}

\item[] {\emph{Lesson Maintainer} for \href{http://www.datacarpentry.org/lessons/}{Geospatial Data Analysis with R} materials}
        \hfill \textbf{2015 to Present}

\item[]
    \begin{innerlist}
      \item Basics of geospatial analysis for research computing. Sep 26 2017, Lawrence Berkeley National Laboratory.
      \item Basics of geospatial analysis for research computing. Jul 21 2016, South Florida Water Management District.
      \item Basics of geospatial analysis for research computing. May 26 2016, South Florida Water Management District.
    \end{innerlist}

\end{outerlist}

\section{Teaching Experience}

\href{https://www.msu.edu/}{\textbf{Michigan State University}},
East Lansing, MI
\vspace{-0.5em}
\begin{outerlist}

\item[] {\emph{Teaching Assistant} for Water in the Environment}
        \hfill \textbf{Spring 2019 to Fall 2019}
\begin{innerlist}
    \item Taught lectures, administered laboratory practicals, marked assignments and exams.
\end{innerlist}
\end{outerlist}

\halfblankline

\href{http://www.utmsi.edu/}{\textbf{University of Texas Marine Science Institute}},
Port Aransas, TX
\vspace{-0.5em}
\begin{outerlist}

\item[] \href{https://doi.org/10.1126/science.331.6021.1127}
        {\emph{National Science Foundation GK\-/12 Graduate Fellow}}
        \hfill \textbf{Fall 2010 to Spring 2011}
\begin{innerlist}
    \item Developed GIS lesson material, delivered lessons, evaluated student work.
\end{innerlist}
\end{outerlist}

\halfblankline

\href{http://utexas.edu}{\textbf{University of Texas}}, Austin, TX
\vspace{-0.5em}
\begin{outerlist}

\item[] \textit{Teaching Assistant} for Introduction to Oceanography
    \hfill \textbf{Fall 2009 to Spring 2010}
    \begin{innerlist}
        \item Taught lectures, administered laboratory practicals, marked assignments and exams.
    \end{innerlist}

\end{outerlist}

\section{Professional Service}

\textbf{Journal Referee}
\begin{innerlist}
    \item \emph{Earth Syst. Sci. Data}, \emph{BioScience}, \href{https://github.com/ropensci/onboarding/issues/118}{\emph{rOpenSci}}, \emph{Texas Water Journal}, \href{https://docs.google.com/spreadsheets/d/1PAPRJ63yq9aPC1COLjaQp8mHmEq3rZUzwUYxTulyu78/edit#gid=856801822}{\emph{Journal of Open Source Software}} [5], \emph{Journal of Atmospheric and Oceanic Technology}, \emph{Ecologial Modelling} [2], \emph{Frontiers in Ecology and Evolution} [2], \href{https://my.peerageofscience.org/peer/JosephStachelek}{\emph{Peerage of Science}}, \href{https://github.com/ReScience/ReScience-submission/pull/50}{\emph{ReScience}}, \emph{Methods in Ecology and Evolution}, \href{https://github.com/openjournals/jose-reviews/issues/42}{\emph{Journal of Open Source Education}}
\end{innerlist}

\halfblankline

\textbf{Other Activities}
\begin{innerlist}
\item Provided comments for: Soranno, P., King, K., Poisson, A., \textbf{Stachelek, J.}, Boudreau, C., Skaff, N., Smith, N. (2017) Cyberinfrastructure support for collaboration and open science in ecology. NSF Request for Information on Future Needs for Advanced Cyberinfrastructure to Support Science and Engineering Research. \url{https://www.nsf.gov/cise/oac/ci2030/pdf/RFI-Soranno-261.pdf}

\item Participated in the NEON spatio-temporal hackathon (2015) - developed tutorials and assessment instruments to teach fundamental big data skills needed to work efficiently with large spatio-temporal data using open tools, such as R and Python. \href{http://www.neoninc.org/updates-events/update/nsf-biocenters-unite-close-scientific-data-skills-gap-focus-phenology}{link}

\end{innerlist}

\halfblankline

\section{Outreach}

\begin{innerlist}

\item \textbf{Stachelek, J.}, Hondula, K., Kincaid, D., Shogren, A., Zwart, J. 2020. Ripples on the web: spreading lake information via Wikipedia. Limnology and Oceanography Bulletin. \href{https://doi.org/10.1002/lob.10382}{10.1002/lob.10382}.

 \item \href{https://en.wikipedia.org/wiki/Wikipedia:WikiProject_Lakes}{WikiProject Lakes}, Apr 07 2017 - Present, \href{https://en.wikipedia.org/wiki/Special:Contributions/Jst4}{Contributions}.

  \item Everglades Day, Feb 20 2016, Guided tours of science activities at the Loxahatchee Impoundment Landscape Assessment, 17th Annual Everglades Day, Loxahatchee National Wildlife Refuge.
  
  \item Interviewed for: National Public Radio, May 25 2016, \href{http://www.npr.org/2016/05/25/477014085/rising-seas-push-too-much-salt-into-the-florida-everglades}{Rising Seas Push Too Much Salt Into the Florida Everglades}.
  
  \item Interviewed for: PBS Newshour, June 10 2015, \href{https://www.youtube.com/watch?v=ggOl-vaXIFk}{Florida's Everglades face new invasive threat: rising sea levels}.

\end{innerlist}

% \section{Other Software Skills}
% \vspace{1em}
% 
% \begin{innerlist}
%   \item Operating Systems: Debian/Ubuntu/Fedora Linux, Windows
%   \item Coding languages: R, Python, Fortran
%   \item Data management: SQLite, netcdf
%   \item Version control systems: Git, GitHub
%   \item Web development: `shiny`, css
%   \item Data science: HPC, AWS, GNU Make
% \end{innerlist}

% \section{Additional Coursework and Training}
% 
% \begin{innerlist}
%         \item Bayesian Statistics Workshop, June 7 2018, Michigan State University, \\ Instructor: Jason Doll.
%         \item Machine Learning Workshop, Nov 27 2017, Global Lakes Ecological Observatory Network, \\ Instructors: Ken Chiu, Luke Winslow, Taylor Leach, Tim Hung. 
%         \item Bayesian Statistics Workshop, June 28 2017, Michigan State University, \\ Instructor: Song Qian.
%         \item Introduction to Bayesian Statistics, Apr 14 2017, Michigan State University, \\ Instructor: Andrew Finley.
%         \item The mgcv Package As a One-Stop-Shop for Fitting Non-Linear Ecological Models, Aug 08 2016, Ecological Society of America. \\ Instructors: Gavin Simpson, David Miller, Eric Pedersen.
%         \item Using R for High Performance Computing, Feb 27 2015, National Institute for Mathematical and Biological Synthesis. \\ Instructor Drew Schmidt.
%         \item Bayesian Inference and Hierachical Modeling, Sep 27 2013, Florida Atlantic University. \\ Instructor: Robert Dorazio.
% \end{innerlist}

\section{Grants}

\begin{innerlist}

\item National Oceanic \& Atmospheric Administration, Florida Sea Grant, \emph{The effects of projected sea-level rise on Everglades coastal ecosystems: Evaluating the potential for and mechanisms of peat collapse using integrated mesocosm and field manipulations.} Assisting Author, PI Tiffany Troxler (FIU) (\$279,216) 

\end{innerlist}

\section{Honors and Awards}

\begin{innerlist}

\item \href{https://grad.msu.edu/fellowships/dissertation}{MSU Dissertation Completion Fellowship}. Summer 2020. 

\item GLEON G21 Student Travel Award. Nov 4 2019. 

\item GLEON G19 Student Travel Award. Nov 27 2017. 

\item Invited participant at the \href{http://unconf17.ropensci.org/}{2017 rOpenSci Conference}. May 25 2017.

\item Finalist for the 2012 NOAA Coastal Management Fellowship for Texas.

\item Best poster award at the 2012 Texas Bays and Estuaries Meeting. Apr 11 2012. \emph{Porewater Water Salinity Dynamics within the Creekbank Areas of an Irregularly Flooded Salt Marsh}. 

\end{innerlist}

\section{Professional Memberships}

% Coastal and Estuarine Research Federation

Association for the Sciences of Limnology and Oceanography

Ecological Society of America

Global Lake Ecological Observatory Network

% Foundation for Open Access Statistics

\end{document}

%%%%%%%%%%%%%%%%%%%%%%%%%% End CV Document %%%%%%%%%%%%%%%%%%%%%%%%%%%%%
